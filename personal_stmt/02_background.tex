
During my junior year of college while studying abroad in Germany, I was given the opportunity to work with Dr. Henry Brighton at the Max-Planck Institute for Human Development - Center for Adaptive Behavior and Cognition in Berlin (MPIB). Our main projects during my six months there involved training various machine learning algorithms to model heuristic decision-making in real world situations. Most notably, we were able to prune a major healthcare questionnaire down to less than 1/10th of its original size while sacrificing less than 5\% of its predictive capabilities in regard to the future healthcare needs of the respondents. In practical terms, the result meant that what began as a lengthy mail questionnaire (60 questions), could now be administered orally by practitioners who could be confident in the survey's predictive capacity in terms of future health risks facing their patients. 

% The experience working with Dr. Brighton provided me with an introduction to research methods in applied computer science that I otherwise wouldn't have had an opportunity for. The mentorship I received was also invaluable. The process taught me several critical lessons: the importance of evaluating a problem from all angles, of collaboration between academia, government, and industry, and the satisfaction of working on problems with the power to have a positive impact on society.

% In my time at MPIB, I also gained an appreciation for Dr. Gerd Gigerenzer's work in risk literacy. What struck me as most impressive was the commitment Dr. Gigerenzer had to bringing consistency to risk communication and educating the public so that people could make more informed choices. A very pronounced example of the misunderstanding and poor communication of risk could be seen in the concurrent downfall of the subprime housing market. 

% This exposure pushed me to want to understand the mechanics behind the economic and financial environments that affect our everyday lives. Even more so, \textbf{the experiences at MPIB inspired in me a belief in the power of true transparency and open collaboration, whether it be in risk communication or software development.} This is a belief that endures in me to this day.

After graduating I found a home for my intellectual curiosities at RPX Corporation in San Francisco. RPX was a young company looking to change and ``de-militarize'' the intellectual property landscape through widespread collaboration between the major developers of new technologies. I began working as part of the Data Science \& Analytics team where the theme of my work was in analyzing large amounts of data to develop insights into the patent market. This included applying machine learning methods to model patent asset valuation and developing a multi-factor approach to predict the likelihood of future occurrence of an infringement lawsuit.

At the start of my second year, I was recruited into the Corporate Development group where I was given the opportunity to develop market strategies cohesive with our own research into modeling and analyzing the particular risks our client companies were facing. Examples include the development of a patent co-defendant insurance offering (built on our predictive infringement modeling) whereby companies could collaboratively take a stand against malicious entities in the patent market. 