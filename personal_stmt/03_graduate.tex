
% Intro and collaboration / interdisciplinary research
Applying to graduate programs in computer science with my set of interests I was particularly interested in programs with significant focus on interdisciplinary research, which is what brought me to Rutgers University. Collaborations within Dr. Kostas Bekris' robotics group include ongoing work with the Psychology department studying human interpretability of robotic movements, human-robot interaction, and crowd simulation projects. Working with the vision groups of Dr. Ahmed Elgammal and Dr. Dimitris Metaxas, I've had opportunity to collaborate with members of the Center for Cognitive Science (RuCCS) on perception-related studies. The support of these groups and opportunities that they provide have been invaluable to me as a student.

% APC
During my first year, our group decided to compete in the Amazon Picking Challenge: a robotics competition with the goal of correctly identifying objects within a warehouse-like shelving unit, grasping a pre-determined but unknown selection of the objects, and placing them securely into an order bin. The project brings together elements of computer vision (perception of objects), motion planning, and mechanical engineering. On the perception task, I worked closely with Dr. Ferreira de Souza, a visiting professor from the Federal University of Espirito Santo. We began with an open-source version of the LINEMOD algorithm for 3D pose estimation and adapted the algorithm to fit the environmentally difficult shelf scenario, where we were able to significantly improve its object detection performance. Additionally, we compiled the largest RGBD database with 6 degree-of-freedom(DoF) ground truth object pose to date which we also freely offered to the community~\cite{rennie2015dataset}.

The challenge provided a unique opportunity to combine research and real-world engineering while building on and learning from the specific expertise of other students. Though the core teams consisted of PhD students in robotic motion planning, manipulation, and perception, the opportunity allowed our team to involve several masters students as well as a handful of undergraduates. All of these students had interest in robotics but no substantial previous experience, and many were able to directly contribute to and take pride in our team's final solution.

%NASA Work
In my second year, my interest in complex robotic systems drew me towards a project with my advisor involving tensegrity robots. While our lab had been successful in the past employing efficient sampling-based motion planning algorithms to the high-dimensional challenge in a physics-based simulation, the problem of planning ``blind'' (i.e., beginning with no prior knowledge) in such a complex system was still very computationally demanding. Having been given the opportunity to spend the summer at NASA ARC, I decided to spend the time working towards more informed solutions to the motion planning problem using Bayesian optimization over previous trajectories. Though the results I was able to achieve were only preliminary, I believe that this direction shows tremendous promise and I was grateful for the opportunity to be able to learn from researchers working on similar problems, and to be able to present my results to them. 