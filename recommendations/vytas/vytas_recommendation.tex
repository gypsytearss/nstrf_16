% Cover letter using letter.sty
\documentclass[12pt]{letter} % Uses 10pt
\usepackage{graphicx}
%Use \documentstyle[newcent]{letter} for New Century Schoolbook postscript font
% the following commands control the margins:
\topmargin=-1in    % Make letterhead start about 1 inch from top of page
\textheight=9in  % text height can be bigger for a longer letter
\oddsidemargin=0pt % leftmargin is 1 inch
\textwidth=6.5in   % textwidth of 6.5in leaves 1 inch for right margin

\begin{document}

%\signature{Dr. Abdeslam Boularias}           % name for signature
\longindentation=0pt                       % needed to get closing flush left
\let\raggedleft\raggedright                % needed to get date flush left

\begin{letter}{}

%\begin{minipage}[b]{0.6\linewidth}
%{\huge\bf Max Planck Institute for\\ Intelligent Systems}
%\end{minipage}
%\hspace{4cm}
\begin{minipage}[b]{0.3\linewidth}
\includegraphics[scale=0.2]{irglogo.png}
\end{minipage}

\medskip\hrule height 1pt
\begin{flushright}
\hfill Vytas SunSpiral\\
\hfill Senior Robotics Researcher \\
\hfill Intelligent Robotics Group - Intelligent Systems Division \\
\hfill NASA Ames Research Center \\
\hfill Email:  vytas.sunspiral@nasa.gov\\
\hfill Phone:  +1 650-604-4363 \\
\vspace{0.5cm}
\hfill Regarding: NASA Space Technology Research Fellowship
\end{flushright}
%\vfill % forces letterhead to top of page

\opening{To Whom it may concern,}

\noindent 

It is my pleasure to write this letter of recommendation in support of the NASA Space Technology Research Fellowship  
application of Colin Rennie, a PhD student in robotics at Rutgers University
under the supervision of a colleague of mine, Dr. Kostas Bekris. 

Colin spent this past summer, the second summer of his graduate studies, under my
supervision at NASA Ames Research Center. In the previous semester before his time here, Colin 
had begun getting up to speed on our ongoing work with tensegrity robotics and 
had shown to his advisor a growing interest in addressing some of the difficult 
problems related to planning for such complex systems. This interest combined with
Colin's aptitude for the subject matter led to his advisor's suggestion that 
Colin spend the summer at Ames under my supervision -- a suggestion that I 
was happy to accept.

% What Colin worked on
Dr. Bekris' group and I have collaborated on projects involving motion planning 
for tensegrity robots for a couple of years now; first through a graduate student 
under his supervision, Zakary Littlefield, and more recently by way of Dr. Bekris' 
Early Career Fellowship award. The progress has gone terrifically, and I'm always 
excited to talk to their group about progress and brainstorm solutions to any 
new problems that arise in their research. In our meetings over the past year or so,
it became clear that their motion planning algorithms can yield very intricate solutions
even in very complex terrains, but that one of the biggest difficulties they faced
was the computation needed to receive these solutions and the resulting time that
computation required. 

This is the problem Colin was excited about addressing. Starting with simpler systems with similar traits,
such as a physics-based snake robot made up of several identical rigid bodies connected in series
and actuated at its joints, Colin's research over the summer focused on using 
machine learning models to determine which combinations of controls for this system 
were likely to yield desirable overall motions for the system. Then instead of searching
the whole space of possible controls randomly, the motion planning algorithm would 
be able to use this model to predict which combinations of controls sent to the system
would result in motion toward the goal and bias the sampling process toward these predictions. 
In my contact with their group since this summer, I've been updated that taking this approach 
to planning with a more complex tensegrity system is now a priority -- and will be a result that I'll 
be excited to see.

% What I know/like about Colin
In my interactions with Colin this summer, he showed himself to be a capable young 
researcher able to take an idea and work through the details and potential difficulties 
with limited supervision. Specifically, I had a very full lab this summer with 
several ongoing projects and not much time to spare, but was happy to see that in 
my update meetings with Colin he would always have made good forward progress, and 
would always be thinking about the potential upcoming difficulties and how to 
solve them. At the end of the summer period, Colin presented his initial findings
to the Intelligent Robotics Group, and I was also impressed by his professionalism 
and overall ability to communicate scientific research well to a group of his peers.

% Conclusion remarks
In conclusion, in the time I've known Colin he has proven himself to be both a very capable
student and a productive and responsible researcher. Furthermore, Colin is someone who draws 
inspiration from various paradigms in his approach to problems -- a desirable trait and
something exciting to see in a young researcher. I was happy to supervise him 
as a visitor to my lab this summer, and I expect very good things from him in the coming years. 

\closing{Sincerely yours,
\\
\fromname{Vytas SunSpiral}
}

\end{letter}
\end{document}
